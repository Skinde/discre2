\documentclass[12pt, twoside]{article}
% Opciones {{{
\usepackage[pdfa, pdfusetitle, unicode=true]{hyperref}
\usepackage[spanish]{babel}
\usepackage[margin=2.5cm, a4paper]{geometry}
\usepackage{luacode}
\usepackage[shortlabels]{enumitem}
\usepackage{import}
\usepackage{xcolor}
\usepackage{fontspec}
\usepackage[mark, raisemark=0.02\paperheight, marknotags]{gitinfo2}
\usepackage{setspace}

\doublespacing

% Btw I use arch
\setmonofont{InconsolataGo Nerd Font}

\newcommand{\btw}{{\color{arch}\texttt{}} }
\newcommand{\git}{{\color{git}\texttt{}} }

\renewcommand{\gitMarkPref}{{\Large\git git}}

% Esto sirve para poner ecuaciones
\usepackage{mathtools}
\usepackage{amssymb}
\usepackage{cancel}
\allowdisplaybreaks

% Esto sirve para poner imágenes{{{
\usepackage{graphicx}
\usepackage{svg}
\usepackage{subcaption}

\usepackage{float}
\usepackage{pgfplots}

\pgfplotsset{compat=1.16}
\graphicspath{ {ima/} }
%}}}
% Colores de los links {{{
\definecolor{red}{HTML}{F22C40}
\definecolor{green}{HTML}{5AB738}
\definecolor{yellow}{HTML}{D5911A}
\definecolor{blue}{HTML}{407EE7}
\definecolor{magenta}{HTML}{6666EA}
\definecolor{cyan}{HTML}{00AD9C}
\definecolor{arch}{HTML}{1793D1}
\definecolor{git}{HTML}{F54D27}

\hypersetup{
	colorlinks=true,
	linkcolor=blue,
	urlcolor=cyan,
	citecolor=magenta,
}
%}}}
% Esto controla a la cabecera {{{
\usepackage{fancyhdr}

\pagestyle{fancy}
\fancyhf{}
\renewcommand{\headrulewidth}{0pt}
\chead{ \textbf{\normalsize{Estructuras discretas II} }}
%\fancyhf[HL]{\includesvg[height=0.8\headheight]{Utec.svg}}
\fancyhf[FL]{\textbf{\thepage}}
\setlength{\headheight}{20pt}
\setlength{\textheight}{675pt}
%}}}
% Título {{{
\title{\textbf{Algo parte 6}}
% Aqui hay que poner a los autores
\author{
		Alberto Oporto Ames\\
		\texttt{alberto.oporto@utec.edu.pe}\\
		%\and <++>\\
		%\texttt{<++>}\\
		}
%}}}
%}}}
% Aquí empieza el documento{{{
\begin{document}
\maketitle
\thispagestyle{fancy}

\begin{align*}
	\vee &\longleftarrow\text{ Join (Menor cota superior)}\\
	\wedge &\longleftarrow\text{ Meet (Mayor cota inferior)}
\end{align*}
\textbf{Un retículo $\mathbf{(A,R)}$ es un $\mathbf{CPO}$ tal que:}
\begin{align*}
	mcs&(\{x,y\}) && \text{Están bien definidas}\\
	MCI&(\{x,y\}) && \text{para todo $x\in A$, $y \in B$}
\end{align*}
\textbf{Abuso: Escribiremos}
$mcs(x,y)$, $MCI(x,y)$ en vez de $mcs(\{x,y\})$, $MCI(\{x,y\})$
y entederemos que si $|\{x,y\}|=1$ la función retorna el valor de la entrada,
es decir se comporta como la identidad sobre conjuntos de tamaño 1.
%\begin{tikzpicture}
	%\draw (0,1) -- (1,0) -- (0,-1) -- (-1,0) -- (0,1);
	%\text{aaaa}(0,1);
%\end{tikzpicture}
\begin{align*}
	D_6 &= \{1,2,3,6\}\\
	%\mathbb{P}_6 &= \{\}\\
\end{align*}
$\Big{\{}\{1\},\{2\},\{3\},\{6\},\{1,2\},\{1,3\},\{2,3\},\{2,6\},\{3,6\}\Big{\}}
=\mathbb{P}_1(A)\cup\mathbb{P}_2(A)$
Debería ser obvio que $mcs(X)$ cuando $X\in \mathbb{P}_1(A)$
está bien definido.
$mcs(X)=X$

Si $(A,R)$ es un retículo eso quiere decir que $mcs(X)\in \mathbb{P}_1(A)\forall X \in
\mathbb{P}_2(A)$

\textbf{Un retículo es acotado}
si existen $\bot , \top \in A$ tales que
\begin{align*}
	\bot Rx \ &\forall x \in A\wedge\\
	xR\top \  &\forall x \in A
\end{align*}

\begin{align*}
	x\vee y &:=mcs(x,y)\\
	x\wedge y &:=MCI(x,y)
\end{align*}
Un retículo es distributivo:
\begin{align*}
	x\vee(y\wedge z) &= (x\vee y)\wedge(x\vee z) \text{\textbf{\ y}}\\
	x\wedge(y\vee z) &= (x\wedge y)\vee (x\wedge z)
\end{align*}

\textbf{Un \underline{retículo es complementado} si:}\\
Para todo $x\in A$ existe (y es único) $y\in A$ tal que
\begin{align*}
	x\wedge y &= \bot\\
	x\vee y &= \top\\
	\Aboxed{y &= \overline{x}} \text{ Complemento de $x$}
\end{align*}

Un $CPO(A,R)$ es una \underline{álgebra booleana}\\
si es un retículo acotado, \underline{distributivo y complementado.}

\textbf{Veamos que } $\Big{(}D_{30},\Big{|}\Big{)}$ es una álgebra booleana.
\begin{figure}[H]
	\centering
	\includesvg[width=0.3\linewidth]{dibujo}
\end{figure}

Vean que $mcs(x,y)=mcm(x,y)$ $MCI(x,y)=MCD(x,y)$\\
Por lo que $\Big{(}D_{30},\Big{|}\Big{)}$
es un retículo, acotado, distributivo, complementado.
$\overline{x} = \frac{30}{x}$ $\Big{(}D_{30},\Big{|}\Big{)}$ es una
algebra booleana.
\begin{align*}
	mcm(x,MCD(y,z)) &= mcm(MCD(x,y),MCS(x,z)) \text{ \textbf{y}}\\
	MCD(x,mcm(y,z)) &= MCD(mcm(x,y),mcm(x,z))
\end{align*}

\begin{align*}
	\forall x,y,z && x \neq y\\
	&& x \neq z\\
	&& y \neq z
\end{align*}
\textbf{Dos caminos:}
\begin{itemize}
	\item Analizar la reacción
	\item Demostrar por fuerza bruta que funciona
		\begin{align*}
			\intertext{¿Es distributivo?:}
			2 \vee (2\wedge 5 ) &= 2 \vee 1 = 2\\
			(2\vee 3) \wedge(2\vee 5) &= 6 \wedge 10 = 2...........
		\end{align*}
\end{itemize}
\textbf{Es $\mathbf{\Big{(}}D_8,\Big{|}\Big){}$ una álgebra booleana}\\
$D_8 = \{1,2,4,8\}$
\begin{figure}[H]
	\centering
	\includesvg[width=0.4\linewidth]{dibujo2}
\end{figure}
Claramente tenemos un retículo\textcolor{red}{, acotado}\textcolor{blue}{, distributivo}
\textcolor{green}{no es complementario}
\begin{align*}
	4\vee(2\wedge 1) &= 4\vee 1 =4\\
	(4\vee 2)\wedge(4\vee1) &= 4\wedge4\\
	2\vee(4\wedge1) &= 1\vee 1 = 2\\
	(2\vee4)\wedge(2\vee1) &= 4\wedge2 = 2\\
	1\vee(4\wedge2) &= 1\vee2 = 2\\
	(1\vee4)\wedge(1\vee2) &= 4\wedge2=2
\end{align*}
\textcolor{green}
{
	\begin{align*}
		2 \vee 8 &= 8 \text{ Único complemento}\\
		\intertext{pero} \cancel{ 2\wedge 8 = 2 }\neq 1\\
		\text{2 no tiene complemento}
	\end{align*}
}
\textbf{¿Es $\mathbf{\Big{(}}\mathbb{P}([3]),\sqsubseteq\Big{)}$ una álgebra booleana?}
( $\sqsubseteq\ \longleftarrow$ subconjunto )\\
\textbf{¿Es $\mathbf{\Big{(}}D_{12},\Big{|}\Big{)}$ una álgebra booleana?}
\begin{figure}[H]
	\centering
	\includesvg[width=0.5\linewidth]{dibujo3}
\end{figure}
\begin{align*}
	mcs(x,y) &= x\cup y\\
	MCI(x,y) &= x\cap y\\
\end{align*}
Es un retículo\textcolor{red}{, acotado}\textcolor{blue}{, distributivo}
\textcolor{green}{, complementado $\overline{x}=[3]-x$}
\textcolor{blue}
{
	\begin{align*}
		\forall x,y,z\\
		x\cap(y\cup z) &= (x\cap y)\cup(x\cap z)\ \text{y}\\
		x\cup(y\cap z) &= (x\cup y )\cap(x\cup z)
	\end{align*}
}
\section*{Tarea}%
\label{sec:Tarea}

¿Qué propiedades hereda meet y join cuando es álgebra booleana?

\end{document}
%}}}
