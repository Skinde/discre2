\documentclass[../main.tex]{subfiles}

\graphicspath{{ima/clase11}{ima}}

% Aquí empieza el documento{{{
\begin{document}

\chapter*{Clase 11}%

\thispagestyle{fancy}

\definicion
Conteo: Si $A$ y $B$ son finitos:
\begin{itemize}
	\item \dobledef{$|A\times B|=|A|*|B|$}{Principio de multiplicación} \textbf{Regla del producto}
	\item \dobledef
		{
			\begin{align*}
				\text{Si }A\cap B &= \cancel{O}\\
				|A\cup B| &= |A|+|B|
			\end{align*}
		}
		{Principio de la suma} \textbf{Regla de la suma}
	\item
		\begin{align*}
			f&:A \longrightarrow B && \text{$f$ total sobreyectiva}\\
			|A| &= k|B| && \text{y $k:1$ $k\in\mathbb{N}$}\\
			\frac{|A|}{k} &= |B|
		\end{align*}
		\dobledef{}{Principio de división}
		\textbf{Regla del cociente}
\end{itemize}

¿Qué pasa si $A\cap B \neq \cancel{O}$?

\textcolor{red}
{
	\begin{align*}
		A\cap B \neq \cancel{O} \Rightarrow\\
		\exists x \in \Omega:x\in A \wedge x \in B
	\end{align*}
}
\begin{align*}
	|A\cup B| &= \overbrace{|A-B|+|A\cap B|}^{|A|}+|B-A|\\
	\\
	|A\cup B| &= |A| + |B-A|\\
	|A\cup B| &= |A-B|+|B|\\
	|A\cup B| &= |A|+|B|- |A\cap B|
\end{align*}
\textbf{\textcolor{red}{Demostración:}}
\begin{align*}
	(A-B)\cup(A\cap B) &= A\\
	|A| &= |A-B|+||A\cap B|\\
	|B| &= |B-A|+||B\cap A|\\
	\\
	A\cup(B-A) &= A\cup B\\
	A\cup(B-A) &= \cancel{O}\\
	\\
	|A|+|B-A| &= |A\cup B|\\
	|B|+|A-B| &= |B\cup A|
\end{align*}

\begin{align*}
	A\cup B &= A \cup(B-A)\\
	|A\cup B| &= |A| + |B-A|\\
	|A\cup B| &= |A| + (|B|-|A\cap B|)\\
	|A\cup B| &= |A| + |B| - |A\cap B|
\end{align*}

\definicion
\textbf{Principio de inclusión-exclusión}\\
$|A|,|B|<\infty$
\[
	|A\cup B| = |A|+|B|-|A\cap B|
\]

\subsection*{Problema:}%

En el menú \underline{DUOLINGO} el dueño dice que puede comer distinto todos los días
del años.

El menú tiene:
\begin{itemize}
	\item 3 entradas
	\item 5 fondos
	\item 7 postres
	\item 2 bebidas
\end{itemize}
\begin{enumerate}
	\item Si un almuerzo consiste en tomar una entrada, un fondo, un postre,
y una bebida.
¿Es verdadera la afirmación del dueño?

\textbf{\textcolor{red}{Regla del producto:\[3*5*7*2=210 < 366\]}}

\textbf{La afirmación es mentira.}

\item Si un almuerzo puede consistir en lo que definimos antes
o una cosa donde falte alguno de los componentes.
¿Es verdad la afirmación?
\[
	\overbrace{(3+1)+(5+1)+(7+1)+(2+1)-1}^
	{\substack{\text{Una opción más: nada.}\\
			\text{Pero $4$ nadas no cuenta como menú}
	}}=575>366
\]
\textbf{Ahora la afirmación es verdadera.}

\item La sunat define \textbf{Almuerzo} como comer
al menos un fondo y bebida.
\[
	(3+1)*(5)*(7+1)*(2) = 320
\]

\textbf{Preso por falsa publicidad}

\item Si se define que el almuerzo debe tener al menos un fondo
y una bebida, una o dos entradas y postre o no postre.

\end{enumerate}

Entradas:
\[
	\underbrace{3+\Bigg{(}\overbrace{\binom{3}{2}}^{\substack{\text{Entradas}\\\text{distintas}} }
	+\overbrace{\binom{3}{1}}^{\substack{\text{Entradas}\\\text{iguales}} }\Bigg{)}}_
	{\text{Entrada}}
\]

Fondo:
\[
	\binom{5}{1}
\]

Postre:
\[
	\Bigg{(}
	\overbrace{\binom{7}{1}}^{\text{$1$ postre}}+
	\overbrace{\binom{7}{0}}^{\text{Ningún postre}}
	\Bigg{)}
\]

Bebida:
\[
	\binom{2}{1}
\]
\[
	\boxed{(3+(3+3))*5*(7+1)*2=720}
\]
{\Huge
	\[
		\binom{n}{k} = \binom{n}{n-k}
	\]
}
\begin{align}
	\intertext{Demostración aburrida:}
	\binom{n}{k} &= \frac{n!}{k!(n-k)!} \\
	\binom{n}{n-k} &= \frac{n!}{(n-k)!(n-(n-k))!} = \frac{n!}{(n-k)!k!}
\end{align}

\begin{align*}
	\Omega \quad &A\subseteq\Omega\\
	\overline{A} &= \Omega-A\\
	\overline{(.)}&:\mathbb{P}(\Omega)\rightarrow\mathbb{P}(\Omega)\\
	\overline{A} &= \Omega -A
\end{align*}
$\overline{(.)}$ Es:
\begin{itemize}
	\item Total
	\item Inyectiva
	\item Sobreyectiva
\end{itemize}
Por lo tanto es $|:|$

$\binom{n}{k}=$ número de formas de hacer un conjunto con $k$ elementos
de un universo con $n$ elemento.

$\binom{n}{n-k}= $ número de formas de hacer un conjunto de $n-k$ elementos
de un universo de $n$ elementos.

\dobledef{El lado izquierdo selecciona directamente $k$ cosas distintas sin orden
de un universo con $n$}
{El lado derecho selecciona $n-k$ cosas del universo de $n$ que serán descartadas
para formar el conjunto de $k$ cosas.}

\begin{itemize}
	\item ¿De cuantás formas podemos seleccionar $2$ colores de $5$?
		\[\binom{5}{2}=10\]
	\item ¿De cuantás formas podemos seleccionar $5$ colores de $2$?
		\[\binom{2}{5}=10\]
\end{itemize}
\begin{figure}[H]
	\centering
	\includesvg[width=0.4\linewidth]{dibujo}
\end{figure}
\[
	\binom{n}{k}= 0 \text{ Si } k>n\wedge k<0
\]
\[
	\binom{n+1}{k}=\binom{n}{k}+\binom{n}{k-1}
\]
El lado izquierdo cuenta el número de formas de seleccionar
$k$ de un universo de $n+1$ cosas diferentes.
\textbf{Puedo decir que una de esas cosas es especial. (EL HUEVO PODRIDO)}

{\Huge
	\[
		\underbrace
		{
			\underbrace{\binom{n}{k}}_
			{\substack{\text{Contar conjuntos}\\
					\text{de tamaño}\\
					\text{$k$ que no}\\
					\text{tienen al}\\
			\text{huevo podrido} }}
			+\underbrace{\binom{n}{k-1}}_
			{\substack{\text{Contar conjuntos}\\
					\text{de tamaño}\\
					\text{$k$ que \textbf{sí}}\\
					\text{tienen al}\\
			\text{huevo podrido} }}
		}_
		{\substack{\text{Cualquier conjunto de tamano $k$ extraído}\\
				\text{del universo de $n+1$ cosas o tiene}\\
		\text{al huevo podrido o no tiene al huevi podrido.} }}
	\]
}

Sabemos que hay $2^n$
\[
	f:[n]\longrightarrow\mathbb{B}
\]
funciones booleanas \underline{totales} diferentes

\textcolor{red}{ \textbf{¿Cuántas funciones de la forma $\mathbf{g:[n]
\longrightarrow\mathbb{B}}$ hay?}}
\end{document}
%}}}
